\documentclass[a4paper, 11pt]{article}
\author{Kajetan Kaczmarek}
\usepackage{amsmath}
\usepackage{graphicx}
\usepackage{listings}
\usepackage[T1]{fontenc}
\usepackage[utf8]{inputenc}
\usepackage[polish]{babel}
\usepackage{color} %red, green, blue, yellow, cyan, magenta, black, white
\definecolor{mygreen}{RGB}{28,172,0} % color values Red, Green, Blue
\definecolor{mylilas}{RGB}{170,55,241}
\graphicspath{{Resources/}} %Setting the graphicspath
\usepackage{float}
\usepackage[caption = false]{subfig}

\lstset{language=Matlab,%
    %basicstyle=\color{red},
    breaklines=true,%
    morekeywords={matlab2tikz},
    keywordstyle=\color{blue},%
    morekeywords=[2]{1}, keywordstyle=[2]{\color{black}},
    identifierstyle=\color{black},%
    stringstyle=\color{mylilas},
    commentstyle=\color{mygreen},%
    showstringspaces=false,%without this there will be a symbol in the places where there is a space
    basicstyle = \tiny,%
    numbers=left,%
    numberstyle={\tiny \color{black}},% size of the numbers
    numbersep=9pt, % this defines how far the numbers are from the text
    emph=[1]{for,end,break},emphstyle=[1]\color{red}, %some words to emphasise
    %emph=[2]{word1,word2}, emphstyle=[2]{style},    
}



\begin{document}
\title{Sprawozdanie STP \\* Projekt nr.2 \\* 
Zadanie 2.19 \\*}
\maketitle
\begin{enumerate}
\item Obiekt\\*
Dany jest obiekt regulacji opisany transmitancją : \[ G(s) = \dfrac{K_0e^{-T_0s}}{(T_1s + 1)(T_2s + 1)}\] %\[ G(s) = \dfrac{K_0e^{-T_0s}}{T_1T_2s^2 + (T_1+T_2)s +1}\]
Gdzie \( K_0 = 2.7 , T_0=5,T_1 = 1.83 , T_2=5.31 \)
\item Obliczanie transmitancji dyskretnej G(z) \\*
Na początek wprowadziłem transmitancję ciągłą do matlab'a za pomocą komendy tf z opcją 's', zgodnie z dokumentacją mathworks.com (\("\) s = tf('s') to specify a TF model using a rational function in the Laplace variable, s\("\)).Następnie zastosowałem procedurę obliczania transmitancji dyskretnej wg. s. 99 ze skryptu do przedmiotu, używając komendy c2d z opcją 'zoh', czyli ekstrapolator zerowego rzędu. Za okres próbkowania przyjąłem podane w zadaniu 0.5 s.Wyliczona transmitancja wyniosła \[ G(z) =z^{-10}\dfrac{0.03078 z^{-1} + 0.02723z^{-2}}{ 1 - 1.671 z^{-1} + 0.6925z^{-2}} \]
Współczynnik wzmocnienia statycznego transmitancji wynosi (skrypt s. 80)
\begin{itemize}
\item Dla transmitancji ciągłej \[K_{stat} = \lim_{s\to 0} G(s) = \lim_{s\to 0}  \dfrac{K_0e^{-T_0s}}{(T_1s + 1)(T_2s + 1)} = 2.7\]

\item Dla transmitancji ciągłej \[K_{stat} = \lim_{z\to 1} G(z) = \lim_{z\to 1}  z^{-10}\dfrac{0.03078 z^{-1} + 0.02723z^{-2}}{ 1 - 1.671 z^{-1} + 0.6925z^{-2}}  \approx 2.7\]
\item Jak widać wzmocnienia są tożsame co potwierdza równoważność postaci ciągłej i dyskretnej. Podobny wniosek można wyciągnąć na podstawie wykresu odpowiedzi skokowej, przedstwawionego poniżej
\end{itemize}
\includegraphics[width=\textwidth, height=\textheight, keepaspectratio]{P1.png}
Kod programu : \\*
\lstinputlisting{P1.m}

\item Obliczenie równania różnicowego , wg. wzoru \[ y(k) = \sum_{i=1}^{n}b_{i}y(k - i) + \sum_{i=1}^{m}c_{i}u(k - i) \] Przekształcając transmitancję dyskretną wg. wzoru \( G(z) = \dfrac{Y(z)}{U(z)}\) otrzymujemy \[ z^{-10}(0.03078 z^{-1} + 0.02723z^{-2})U(z) =  ( 1 - 1.671 z^{-1} + 0.6925z^{-2})Y(z)\]
Co daje nam 
\[ 0.03078 z^{-11}U(z)  + 0.02723z^{-12}U(z) =   Y(z) - 1.671 z^{-1}Y(z) + 0.6925z^{-2}Y(z)\]
I po zastosowaniu odwrotnej transformaty z oraz uporządkowaniu : 
\[y(k) =  1.671 y(k-1) - 0.6925y(k-2) + 0.03078 u(k-11)  + 0.02723u(k-12)    \]
Przy obliczeniach podpierałem się materiałem ze stron 87-89 skryptu do przedmiotu
\item Dobranie regulatora PID metodą Zieglera - Nicholsa \\*\\*
Podane wartości : \( K_r = 0.6K_k, T_i = 0.5T_k, T_d = 0.12T_k  \)
\\*\\*
Metoda Zieglera - Nicholsa to metoda heurystyczna oparta o pomiar parametrów oscylacji. Polega na stopniowym zwiększaniu oscylacji aż osiągnie się wzmocnienie \( K_u\) dla którego sygnał oscyluje ze stałą amplitudą(tzw. wzmocnienie krytyczne).Następnie używamy tabeli ze stałymi aby dobrać parametry regulatora PID.

\begin{center}
\begin{tabular}{|c|c|c|c|} \hline
 & $K$ & $T_i$ & $T_d$ \\
\hline 
P & $0,5K_u$ & - & -\\
\hline
PI & $0,45K_u$ & $0,83T_u$ & - \\
\hline
PID & $0,6K_u$ & $0,5T_u$ & $0,12T_u$ \\
\hline 
\end{tabular}\\
\end{center}

\begin{figure}
\subfloat[fig 1]{\includegraphics[width=3in, height=\textheight, keepaspectratio]{{P2_K0.1}.png}}
\subfloat[fig 2]{\includegraphics[width=3in, height=\textheight, keepaspectratio]{{P2_K0.15}.png}}\\
\subfloat[fig 3]{\includegraphics[width=3in, height=\textheight, keepaspectratio]{{P2_K0.2}.png}}
\subfloat[fig 4]{\includegraphics[width=3in, height=\textheight, keepaspectratio]{{P2_K0.25}.png}}\\
\subfloat[fig 5]{\includegraphics[width=3in, height=\textheight, keepaspectratio]{{P2_K0.3}.png}}
\subfloat[fig 6]{\includegraphics[width=3in, height=\textheight, keepaspectratio]{{P2_K0.35}.png}}
\end{figure}


\begin{figure}
\subfloat[fig 7]{\includegraphics[width=3in, height=\textheight, keepaspectratio]{{P2_K0.4}.png}}
\subfloat[fig 8]{\includegraphics[width=3in, height=\textheight, keepaspectratio]{{P2_K0.45}.png}}\\
\subfloat[fig 9]{\includegraphics[width=3in, height=\textheight, keepaspectratio]{{P2_K0.5}.png}}
\subfloat[fig 10]{\includegraphics[width=3in, height=\textheight, keepaspectratio]{{P2_K0.55}.png}}\\
\subfloat[fig 11]{\includegraphics[width=3in, height=\textheight, keepaspectratio]{{P2_K0.6}.png}}
\subfloat[fig 12]{\includegraphics[width=3in, height=\textheight, keepaspectratio]{{P2_K0.65}.png}}
\end{figure}

\begin{figure}
\subfloat[fig 13]{\includegraphics[width=3in, height=\textheight, keepaspectratio]{{P2_K0.7}.png}}
\subfloat[fig 14]{\includegraphics[width=3in, height=\textheight, keepaspectratio]{{P2_K0.75}.png}}\\
\subfloat[fig 15]{\includegraphics[width=3in, height=\textheight, keepaspectratio]{{P2_K0.8}.png}}
\subfloat[fig 16]{\includegraphics[width=3in, height=\textheight, keepaspectratio]{{P2_K0.85}.png}}\\
\subfloat[fig 17]{\includegraphics[width=3in, height=\textheight, keepaspectratio]{{P2_K0.9}.png}}
\subfloat[fig 18]{\includegraphics[width=3in, height=\textheight, keepaspectratio]{{P2_K0.95}.png}}

\end{figure}
\newpage
Metodą prób i błędów doszedłem do wartości K = 0.815 , okres oscylacji \(T_u = 18 s\)\\
\includegraphics[width=\textwidth, height=\textheight, keepaspectratio]{{P2_K0.815}.png}
\\*
Po podstawieniu daje nam to \[ K_p = 0.489\]\[ T_i = 9 \]\[T_d = 2,25\]
Kod programu : \\*
\lstinputlisting{P3_ZN.m}
 Regulator PID jest opisany wzorem \[ u(t) = (u_p(t) + u_I(t) + u_D(t))e(t) = K(1+\dfrac{1}{T_i}\int_0^te(\tau)d\tau + T_d\dfrac{de(t)}{dt})  e(t) \]
 oraz w wersji dyskretnej 
 \[ u(k) = \dfrac{r_2z^{-2} + r_1z^{-1} + r_0}{1-z^{-1}}e(k) \]
 gdzie
 \[r_2 = \dfrac{KT_d}{T} = 2.2005\]
 \[r_1 = K(\dfrac{T}{2T_i} -\dfrac{2T_d}{T} - 1   )= - 4.876416 \]
 \[r_0 = K(1 + \dfrac{T}{2T_i} +\dfrac{T_d}{T})= 2.70308 \]
\item Testowanie regulatora PID \\*
Symulacja otrzymanego regulatora : \\*
\includegraphics[width=\textwidth, height=\textheight, keepaspectratio]{{P4_1_r12.2005r2-4.8764r32.7031}.png}

Regulator jest przeregulowany : Wymagane jest ręczne dostrojenie.
Po ręcznym dostrojeniu : \\*

\includegraphics[width=\textwidth, height=\textheight, keepaspectratio]{P4_2.png}
 \[r_2 =     -1.8400 \]
 \[r_1 = 3.2264\]
 \[r_0 =    -1.3736 \]
Kod Programu do testowania : \\*
\lstinputlisting{P4_PID.m}
 \item Regulator DMC \\*
 Kod Programu : \\*
 \lstinputlisting{P4_DMC_Init.m}
 
 Symulacje wstępne dla rosnących wartości D, N , Nu \\*

\begin{figure}
\subfloat[fig 1]{\includegraphics[width=3in, height=\textheight, keepaspectratio]{{P4_DMC_D_15}.png}}
\subfloat[fig 2]{\includegraphics[width=3in, height=\textheight, keepaspectratio]{{P4_DMC_D_20}.png}}\\
\subfloat[fig 3]{\includegraphics[width=3in, height=\textheight, keepaspectratio]{{P4_DMC_D_25}.png}}
\subfloat[fig 4]{\includegraphics[width=3in, height=\textheight, keepaspectratio]{{P4_DMC_D_30}.png}}\\
\subfloat[fig 5]{\includegraphics[width=3in, height=\textheight, keepaspectratio]{{P4_DMC_D_35}.png}}
\subfloat[fig 6]{\includegraphics[width=3in, height=\textheight, keepaspectratio]{{P4_DMC_D_40}.png}}
\end{figure}

\begin{figure}
\subfloat[fig 7]{\includegraphics[width=3in, height=\textheight, keepaspectratio]{{P4_DMC_D_45}.png}}
\subfloat[fig 8]{\includegraphics[width=3in, height=\textheight, keepaspectratio]{{P4_DMC_D_50}.png}}\\
\subfloat[fig 9]{\includegraphics[width=3in, height=\textheight, keepaspectratio]{{P4_DMC_D_55}.png}}
\subfloat[fig 10]{\includegraphics[width=3in, height=\textheight, keepaspectratio]{{P4_DMC_D_60}.png}}\\
\subfloat[fig 11]{\includegraphics[width=3in, height=\textheight, keepaspectratio]{{P4_DMC_D_65}.png}}
\subfloat[fig 12]{\includegraphics[width=3in, height=\textheight, keepaspectratio]{{P4_DMC_D_70}.png}}
\end{figure}
\newpage
Po seri testów ustaliłem N = \( N_u \) = 55


\begin{figure}
\subfloat{\includegraphics[width=3in, height=\textheight, keepaspectratio]{{P4_DMC_D_55N_Nu_55}.png}}
\subfloat{\includegraphics[width=3in, height=\textheight, keepaspectratio]{{P4_DMC_D_50N_Nu_55}.png}}\\
\subfloat{\includegraphics[width=3in, height=\textheight, keepaspectratio]{{P4_DMC_D_45N_Nu_55}.png}}
\subfloat{\includegraphics[width=3in, height=\textheight, keepaspectratio]{{P4_DMC_D_40N_Nu_55}.png}}\\
\subfloat{\includegraphics[width=3in, height=\textheight, keepaspectratio]{{P4_DMC_D_35N_Nu_55}.png}}
\subfloat{\includegraphics[width=3in, height=\textheight, keepaspectratio]{{P4_DMC_D_30N_Nu_55}.png}}
\end{figure}
Z ww. symulacji wynika że dobrym punktem granicznym jest D = 35 \\*
Kod Programu : \\*
 \lstinputlisting{P5_DMC_D.m}
\newpage
Kolejna seria testów przy ustalonym D = 35

Po testowaniu różnych wartości Nu wyniki sugerują że przy przyjętych przeze mnie pozostałych parametrach zmiana w zakresie do 5 przynosi widoczne zmiany, natomiast następnie następuje ujednolicenie wykresów

\begin{figure}
\subfloat{\includegraphics[width=3in, height=\textheight, keepaspectratio]{{P4_DMC_Nu_1}.png}}
\subfloat{\includegraphics[width=3in, height=\textheight, keepaspectratio]{{P4_DMC_Nu_2}.png}}\\
\subfloat{\includegraphics[width=3in, height=\textheight, keepaspectratio]{{P4_DMC_Nu_3}.png}}
\subfloat{\includegraphics[width=3in, height=\textheight, keepaspectratio]{{P4_DMC_Nu_4}.png}}\\
\subfloat{\includegraphics[width=3in, height=\textheight, keepaspectratio]{{P4_DMC_Nu_5}.png}}
\subfloat{\includegraphics[width=3in, height=\textheight, keepaspectratio]{{P4_DMC_Nu_6}.png}}
\end{figure}

\begin{figure}
\subfloat{\includegraphics[width=3in, height=\textheight, keepaspectratio]{{P4_DMC_Nu_10}.png}}
\subfloat{\includegraphics[width=3in, height=\textheight, keepaspectratio]{{P4_DMC_Nu_15}.png}}\\
\subfloat{\includegraphics[width=3in, height=\textheight, keepaspectratio]{{P4_DMC_Nu_20}.png}}
\subfloat{\includegraphics[width=3in, height=\textheight, keepaspectratio]{{P4_DMC_Nu_25}.png}}\\
\end{figure}
\newpage

Kod Programu : \\*
 \lstinputlisting{P5_DMC_Nu.m}

Testowanie wartości Lambdy w zakresie <1,6> dla N= 55 , Nu = 5, D = 35
\begin{figure}
\subfloat{\includegraphics[width=3in, height=\textheight, keepaspectratio]{{P4_DMC_Lambda_1}.png}}
\subfloat{\includegraphics[width=3in, height=\textheight, keepaspectratio]{{P4_DMC_Lambda_2}.png}}\\
\subfloat{\includegraphics[width=3in, height=\textheight, keepaspectratio]{{P4_DMC_Lambda_3}.png}}
\subfloat{\includegraphics[width=3in, height=\textheight, keepaspectratio]{{P4_DMC_Lambda_4}.png}}\\
\subfloat{\includegraphics[width=3in, height=\textheight, keepaspectratio]{{P4_DMC_Lambda_5}.png}}
\subfloat{\includegraphics[width=3in, height=\textheight, keepaspectratio]{{P4_DMC_Lambda_6}.png}}
\end{figure}
\newpage

Kod Programu : \\*
 \lstinputlisting{P5_DMC_Lambdam.m}
 Z symulacji wynika że lambda na poziomie 2~3 osiąga pożądaną maksymalną dokładność

\item Porównanie PID i DMC\\*
Kod Programu : \\*
 \lstinputlisting{P6.m}
\includegraphics[width=\textwidth, height=\textheight, keepaspectratio]{P6_PIDvsDMC.png}
Porównując algorytmy możemy stwierdzić wyższość odpowiednio dobranego algorytmu DMC który szybciej osiąga wartość zadaną, i pomimo gwałtowniejszych ingerencji nie przekracza odchyłu systemu PID \\*
\newpage Wykres zależności \( K_0/K_0^{nom}\) od \(T_0/T_0^{nom}\) \\*
\includegraphics[width=\textwidth, height=\textheight, keepaspectratio]{P6_K.png}
\end{enumerate}
\end{document}

